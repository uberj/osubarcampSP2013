\documentclass[10pt]{beamer}
\usepackage[T1]{fontenc}
\usepackage{soul}
\usepackage{cmap}
\usepackage[utf8]{inputenc}
\usepackage{listings}
\usepackage{courier}
\DeclareGraphicsExtensions{.png}

% Custom colors
\usepackage{color}
\definecolor{deepblue}{rgb}{0,0,0.5}
\definecolor{deepred}{rgb}{0.6,0,0}
\definecolor{deepgreen}{rgb}{0,0.5,0}
\definecolor{grey}{gray}{0.9}

% Python style for highlighting
\newcommand\pythonstyle{\lstset{
    language=Python,
    tabsize=2,
    breaklines=true,
    showstringspaces=false
    backgroundcolor=\color{grey},
    basicstyle=\ttfamily\scriptsize,
    keywordstyle=\color{deepblue}\ttfamily,
    stringstyle=\color{deepred}\ttfamily,
    commentstyle=\color{deepgreen}\ttfamily,
}}

\newcommand\shellstyle{\lstset{
    tabsize=2,
    breaklines=true,
    showstringspaces=false
    backgroundcolor=\color{grey},
    basicstyle=\ttfamily\scriptsize,
}}
% Shell environment
\lstnewenvironment{shell}[1][]
{
\shellstyle
\lstset{#1}
}
{}

% Python environment
\lstnewenvironment{python}[1][]
{
\pythonstyle
\lstset{#1}
}
{}

% Python for external files
\newcommand\pythonexternal[1]{{
\pythonstyle
\lstinputlisting{#1}}}

% Python for inline
\newcommand\pythoninline[1]{{\pythonstyle\lstinline!#1!}}


\begin{document}

\title{Searching Inventory}
\author{Jacques Uber}
\date{\today}

\frame{\titlepage}
\section{Searching}
\frame{\frametitle{What is Inventory?}
\begin{itemize}
    \item Infrastructure information
        \begin{itemize}
            \item Servers
            \item DNS Records
            \item Datacenters
            \item Racks
            \item VLANS
            \item Networks, etc...
            \item etc...
        \end{itemize}
    \item A good search helps a sysadmin troubleshoot
\end{itemize}
}


\frame{\frametitle{Example Questions}
\begin{itemize}
    \item "Where is server X?"
    \item "What is in the 10.2.4.0/24 network?"
    \item "Show all the servers with 'hp' and 'mozilla.com' in their name."
    \item "Show all the servers and DNS records that match 'foo{05-10}.mozilla.com'. Only visible from the Internet?"
    \item "Show all 'AAAA' records in the zone 'foo.mozilla.com' that also have 'firewall' in their name."
\end{itemize}
    ...many more
}

\frame{\frametitle{Possible Search Solutions}
\begin{enumerate}
    \item Giant case statment
        \begin{itemize}
            \item Checkboxes selecting which types to search
            \item Multiple search fields
            \item Joins?
        \end{itemize} \pause
    \item A search DSL (Domain Specific Language)
        \begin{itemize}
            \item keywords (Think Google search)
            \item Compile query string to SQL
            \item Let the database do the hard work
            \item What this talk is about.
        \end{itemize}
\end{enumerate}
}

\frame{\frametitle{My Goal}
\begin{enumerate}
    \item Search everything from one place \pause
        \begin{itemize}
            \item Every \emph{searchable} object should expose properties to be searched \pause
        \end{itemize}
    \item Custom searches defined in the application \pause
    \item "Fast" \pause
        \begin{itemize}
            \item Number of Inventory users usually very low. Doesn't need to scale \pause
        \end{itemize}
    \item Reusable
\end{enumerate}

}

\section{Inventory DSL}
\frame{\frametitle{Things to consider}
\begin{enumerate}
    \item Everything is stored in a relational database \pause
        \begin{itemize}
            \item Fixed schema \pause
            \item Strongly typed data and not everything is text (I.E. Ip addresses)\pause
            \item In the case of DNS an explicit tree structure \pause
        \end{itemize}
    \item Use django's ORM to talk to the database \pause
    \item Maybe target Django Q objects?
\end{enumerate}
}

\frame{\frametitle{An Example Search Query}
\begin{itemize}
    \item "firewall db" \pause
    \item The user is asking 'Show me everything that matches "firewall" and "db"' \pause
        \begin{itemize}
            \item Note the implicit AND
        \end{itemize}
\end{itemize}
}

\frame{\frametitle{An Example Search Query Cont.}
\begin{itemize}
    \item "firewall OR db" \pause
    \item Show me everything that matches "firewall" or "db" \pause
    \item Explicit OR
\end{itemize}
}

\frame{\frametitle{An Example Search Query Cont.}
\begin{itemize}
    \item "firewall db OR host2 console" \pause
    \item Show me everything that matches "firewall" and "db" or "host2" and "console" \pause
        \begin{itemize}
            \item ('firewall AND 'db') OR ('host2' AND 'console')
        \end{itemize}
\end{itemize}
}

\frame{\frametitle{An Example Search Query Cont.}
\begin{itemize}
    \item "(firewall db OR host2 !console) AND zone=:dev.mozilla.com view=:public" \pause
    \item Show me everything that matches "firewall" and "db" or "host2" and doesn't contain "console", and it must also be a DNS record in the dev.mozilla.com DNS zone, and in the public view. \pause
        \begin{itemize}
            \item Things are getting complex. \pause
            \item This looks like SQL! \pause
            \item Note that this search only returns DNS records \pause
            \begin{itemize}
                \item Some types of objects are not appropriate for certain searches
            \end{itemize}
        \end{itemize}
\end{itemize}
}


\begin{frame}[fragile]
\frametitle{Target Django Querysets}
\begin{itemize}
\item (Deap Breath) Take a step back, start with a query strings with only one term \pause
\item Consider the query string "firewall" \pause
\item Simply return anything that matches "firewall" \pause
\item How would we use a django Queryset to do this? \pause
\end{itemize}

\begin{python}
>>> q = Q(hostname__icontains='firewall')
>>> type(q)
<class 'django.db.models.query_utils.Q'>
>>> System.objects.filter(q)
[<System: firewall.db.mozilla.org>]
>>>
\end{python}

\end{frame}

\begin{frame}[fragile]
\frametitle{Target Django Querysets Cont.}
\begin{itemize}
\item We want to search more than one column (or model attribute) \pause
\item For example, \pythoninline{class System} has many attributes that might match a given search term \pause

\item Search all attributes of \pythoninline{class System} and filter on the text "firewall": \pause
\end{itemize}
\begin{python}
    def build_filter(filter_value, fields):
        final_filter = Q()
        for field in fields:
            final_filter = final_filter | Q(
                **{"{0}__icontains".format(field): filter_value}
            )
        return final_filter

    System.search_fields = ['hostname', 'oob_ip', ... ]
    q = build_filter('firewall', System.search_fields)

    System.objects.filter(q)

\end{python}
\end{frame}

\begin{frame}[fragile]
\frametitle{Target Django Querysets Cont.}
\begin{itemize}
\item A goal is to search everything \pause
\item Map \pythoninline{build_filter} over every \emph{searchable} type \pause
    \begin{itemize}
        \item Use each model's \pythoninline{search_fields} attribute to do this dynamically. \pause
    \end{itemize}
\end{itemize}
\begin{python}
searchables = (
    ('A', AddressRecord),
    ('CNAME', CNAME),
    ('MX', MX),
    ('SYSTEM', System),
    ...
    ...
)

def compile_Q(filter_value):
    result = []
    for name, Klass in searchables:
        result.append(build_filter(filter_value, Klass.search_fields))
    return result
\end{python}
\end{frame}

\begin{frame}[fragile]
\frametitle{Target Django Querysets: Operators}
\begin{itemize}
\item \pythoninline{compile_Q} returns a list of querysets that can be used to filter Django models (the classes contained in \pythoninline{searchables})\pause
\item We want to logically combine these lists of querysets \pause
\item Think vector operations: \pause
\end{itemize}
$$
\begin{bmatrix} result_1 \\ result_2 \\ ... \\ result_n \end{bmatrix} =
\begin{bmatrix} q_1 \\ q_2 \\ ... \\ q_n \end{bmatrix} \&
\begin{bmatrix} p_1 \\ p_2 \\ ... \\ p_n \end{bmatrix}
$$

Supported Q operators:
\begin{itemize}
    \item AND (\&)
    \item OR (|)
    \item NOT (\textasciitilde)
\end{itemize}
\end{frame}

\begin{frame}[fragile]
\frametitle{Operators: An Example}
\begin{itemize}
    \item "host1 OR ( firewall AND db )" \pause
\end{itemize}
\begin{python}
def OR(left_q_sets, right_q_sets):
    result = []
    for ql, qr in zip(left_q_sets, right_q_sets):
        result.append(ql | qr)
    return result

def AND(left_q_sets, right_q_sets):
    result = []
    for ql, qr in zip(left_q_sets, right_q_sets):
        result.append(ql & qr)
    return result

result = OR(
    compile_Q("host1"),
    AND(compile_Q("firewall"), compile_Q("db"))
)
\end{python}
\end{frame}

\begin{frame}
\frametitle{Front End}
\begin{itemize}
    \item We have the builing blocks (\pythoninline{compile_Q}) and operators to compile queries \pause
    \item We need to interpret user search queries dynamically \pause
    \item It is a solved problem! Yay!
\end{itemize}
\end{frame}

\begin{frame}
\frametitle{Parsley}
\begin{itemize}
    \item Parsley: "Parsley is a pattern matching and parsing tool for Python programmers." \pause
    \item http://parsley.readthedocs.org/
\end{itemize}
\end{frame}

\begin{frame}[fragile]
\frametitle{DSL Part 1}
\begin{python}
ws = ' '*
wss = ' '+
not_ws = :c ?(c not in (' ', '\t')) -> c
letter = :c ?('a' <= c <= 'z' or 'A' <= c <= 'Z') -> c
special = '_' | '.' | '-' | ':' | ','

# Lexical Operators
NOT = '!'
AND = <letter+>:and_ ?(and_ == 'AND') -> self.AND_op()
OR = <letter+>:or_ ?(or_ == 'OR') -> self.OR_op()

# Directive
EQ = '=:'
d_lhs = letter | '_'
d_rhs = letterOrDigit | special | '/'
DRCT = <d_lhs+>:d EQ <d_rhs+>:v -> self.directive(d, v)

# Regular Expression
RE = '/' <(not_ws)+>:r -> self.regexpr(r)

# Regular text
text = (~OR ~AND ~NOT <(letterOrDigit | special )+>:t) -> t
TEXT = <text+>:t -> self.text(t)
\end{python}
\end{frame}

\begin{frame}[fragile]
\frametitle{DSL Part 2}
\begin{python}
# DSF (Device Specific Filters)
DSF = DRCT | RE | TEXT

# An atmon
atom = DSF | parens

value = NOT ws atom:a -> self.NOT_op()(a)
| atom

# Parens
parens = '(' ws expr:e ws ')' -> e

# Operators Precidence
# 1) i_and
# 2) 2_and
# 3) e_or

# x AND y <-- Explicit AND
e_and = AND:op wss value:v -> (op, v)

# x y <-- Implicit AND
i_and = (' '+ ~OR ~AND) value:v -> (self.AND_op(), v)

# x OR y <-- Explicit OR
e_or = OR:op wss expr_2:v -> (op, v)
\end{python}
\end{frame}

\begin{frame}[fragile]
\frametitle{DSL Part 3}
\begin{python}
# Compile
expr = expr_2:left ws e_or*:right -> self.compile(left, right)
expr_2 = expr_3:left ws e_and*:right -> self.compile(left, right)
expr_3 = value:left i_and*:right -> self.compile(left, right)
\end{python}
\end{frame}

\begin{frame}[fragile]
\frametitle{DSL}
Parsley does all the hard stuff: \pause
\begin{shell}
[uberj@box ~]$ python invdsl.py 'firewall (baz OR fiz)'
(firewall AND (baz OR fiz))
\end{shell} \pause

\begin{shell}
[uberj@box ~]$ python invdsl.py 'firewall baz OR fiz'
((firewall AND baz) OR fiz)
\end{shell}
\end{frame}

\begin{frame}
\frametitle{So Far...}
\begin{itemize}
    \item The search is kind of like the SELECT clause in SQL \pause
    \item Good enough for just searching all your django models \pause
    \item General enough to tack on some more interesting queries
\end{itemize}
\end{frame}

\begin{frame}[fragile]
\frametitle{Directives}
\begin{itemize}
    \item A search term corresponds to a list of Qs that filter \emph{searchable} objects based on that term \pause
    \begin{itemize}
        \item We have only seen filters like: \pause
    \end{itemize}
\end{itemize}
    \begin{python}
            filter_list = [
                ...
                Q(field_name__icontains="keywordstring") | ...
                ...
            ]
    \end{python}
\end{frame}

\begin{frame}[fragile]
\frametitle{Directives Cont.}
\begin{itemize}
    \item The list of Q objects can be built to do a more complex filter \pause
\end{itemize}
\begin{python}
searchables = (
    ('A', AddressRecord),
    ('CNAME', CNAME),
    ('MX', MX),
    ('SYSTEM', System),
    ...
    ...
)

def build_rdtype_qsets(rdtype):
    rdtype = rdtype.upper()
    select = Q(pk__gt=-1)
    no_select = Q(pk__lte=-1)
    result = []
    for name, Klass in searchables:
        if name == rdtype:
            result.append(select)
        else:
            result.append(no_select)
    return result

\end{python} \pause
We can now use \pythoninline{build_type_qset('MX')} to give us a list of Q objects that when mapped onto \emph{searchables} will only return 'MX' records.
\end{frame}

\begin{frame}[fragile]
\frametitle{Directives Cont.}
\begin{itemize}
    \item Filter by DNS view\pause
\end{itemize}
\begin{python}
def build_view_qsets(view_name):
    """Filter based on DNS views."""
    view_name = view_name.lower() # Let's get consistent
    try:
        view = View.objects.get(name=view_name)
    except ObjectDoesNotExist:
        raise BadDirective("'{0}' isn't a valid view.".format(view_name))
    view_filter = Q(views=view) # This will slow queries down due to joins
    q_sets = []
    select = Q(pk__gt=-1)
    for name, Klass in searchables:
        if name == 'SOA':
            q_sets.append(select) # SOA's are always public and private
        elif hasattr(Klass, 'views'):
            q_sets.append(view_filter)
        else:
            q_sets.append(None)
    return q_sets
\end{python}
\end{frame}

\begin{frame}
\frametitle{Making things \st{worse} better!}
    \includegraphics[width=4.3in]{lol_search_dsl}
\end{frame}

\begin{frame}
\frametitle{Demo}
\begin{itemize}
    \item Live Demo!
    \item Let's hope it works!
\end{itemize}
\end{frame}
\end{document}

